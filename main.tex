\documentclass[conference, 10pt]{ieeetran}
%\usepackage{cite}
\usepackage{amsmath, amssymb, amsfonts}
\usepackage{algorithmicx}
\usepackage{graphicx}
\usepackage{textcomp}
\usepackage{xcolor}
\usepackage{subcaption}
\usepackage{flushend}
\usepackage{booktabs}
\usepackage[inline]{enumitem}
\usepackage{nohyperref}
\usepackage{url}
\def\BibTeX{{\rm B\kern-.05em{\sc i\kern-.025em b}\kern-.08em

    T\kern-.1667em\lower.7ex\hbox{E}\kern-.125emX}}

\begin{document}

\title{Vivian: Decentralized Global Naming and Storage System on Tangle}

\author{TIAN Xiangan \\ xtianae@connect.ust.hk}

\maketitle

\begin{abstract}

With the significant adoption of distributed ledger technology (DLT) such as blockchain, many previous IT architectures can have alternative decentralized approaches for more secure, transparent, and immutable data storage.
In this paper, we present the design and implementation of Vivian, a new decentralized global naming and storage system based on IOTA Tangle for re-decentralizing the current Internet service and building decentralized applications.
Unlike the traditional Domain name Service (DNS), trust points like DNS root servers are removed and critical data bindings are secured by the distributed ledger. All the nodes in the system form a peer-to-peer network for data sharing.
The P2P network is established through Kademlia DHT, mDNS peer discovery and eventually consistency of data is ensured by Gossip protocol.
In this system, users can own their application data directly rather than relying on the central authorities. The system has no single point failure and the nodes in the network do not need to trust each other.
By using IOTA Tangle, a directed-acyclic-graph (DAG) structure distributed ledger, the system inherits its scalable, lightweight, and feeless characteristics and enables the possibility of application in Internet-of-Thing services.

\end{abstract}

\section{Introduction}
\label{sec:introduction}

A distributed ledger is a type of distributed database that assumes the presence of nodes with malicious intentions.
A distributed ledger comprises a ledger's multiple replications in which data can only bye appended or read.

\end{document}