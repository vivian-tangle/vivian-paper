\section{Applications}
\label{sec:application}

Vivian is designed to be a infrastructure for reconstructing the current Internet services in a decentralized manner.
It is built on Tangle distributed ledger for a better expandability of features in Internet-of-Things Services.
To better understand how Vivian can be implemented in real-world scenarios, we list some potential use cases below:

\begin{itemize}
    \item \textbf{Decentralized Website.} Users can build and maintain their website through Vivian decentralized naming system and storage layer.
          Compared with traditional websites, decentralized websites are censorship-resistance (only the owners can censor or modify the the content of their websites),
          more robust (it is much more difficult to take them down or DDoS), and private (the owner of the decentralized websites can be anonymous).
    \item  \textbf{Identity Attachment.} Users can attach identity information such as GPG public keys, email address, cryptocurrency addresses that are not human-meaningful or easy to memorize to name they like.
    \item \textbf{Enhance IoT Scalability and Privacy.} Inferior scalability, server failure, large-scale data management, and data privacy are four of the weaknesses of the current IoT network implementation \cite{farahani2021convergence}.
          Traditionally, all the data is transmitted from a device or an object to central cloud servers where it is stored and analyzed. If the centralized server fails, the whole network is at risk of taken down.
          As more and more devices joining IoT network, scalability issues are getting worse. Also these devices are producing massive amounts of data including sensitive information, and large-scale data management and data privacy issues are becoming more severe.
          Vivian and IOTA can help to decentralize the current IoT network, and data can be stored on Tangle DL or storage layer provided by Vivian. This improves the scalability and data privacy of IoT services.
\end{itemize}

