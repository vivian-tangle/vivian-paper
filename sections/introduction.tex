\section{Introduction}
\label{sec:introduction}

A distributed ledger is a type of distributed database which tolerates nodes with malicious intentions in the system.
And distributed ledger technology (DLT) enables the realization and operation of distributed ledgers,
which allows benign nodes, to agree on an almost immutable record of transactions with Byzantine failure tolerance (BFT) and eventual consistency via a predefined consensus mechanism \cite{Sunyaev2020}.
Blockchain is one of the most well-known DLTs which was first implemented on Bitcoin. It proposed a simple but robust way for transaction data storage without relying on trust of third parties \cite{nakamoto2008peer}.
Blockchain also ensures improved security and anonymity of Bitcoin transactions compared with traditional electronic transactions.
Since the introduction of Bitcoin in 2009, DLT based cryptocurrencies have made a great impact on financial sectors. Later on people also discovered that the usefulness of DLTs is beyond exchange of currencies and
significant adoption of DLTs were made in many other industries for other different services. Namecoin is the first \texttt{altcoin}\footnote{Altcoin: any cryptocurrencies that are not Bitcoin.} for being the first to create its own blockchain separate from Bitcoin's \cite{kalodner2015empirical}.
And its functionalities are not limited to financial transactions.
The creation of Namecoin was inspired by the idea of BitDNS \cite{merited2010bitdns} and for establishing a decentralized domain name looking up system.


The Internet today is a widespread information infrastructure and its history can date back to 1970s, when \texttt{ARPANET}\footnote{ARPANET: Advanced Research Projects Agency Network. The first wide-area packet-switching network with distributed control originally established by United States Department of Defense.} was developed.
Hosts in the network were assigned names for more convenient use and memorization by humans. With the growth of the network, it became impossible to store all the hosts in a single table.
And Domain Name System (DNS) invented by Paul Mockapetris of USC/ISI permitted a scalable distributed mechanism for resolving hierarchical host names into Internet addresses \cite{leiner2009brief}.
The coordination and management of DNS Root, IP Addressing, and other Internet protocols are in the charge of \texttt{IANA}\footnote{IANA: Internet Assigned Numbers Authority. Website: \url{https://www.iana.org/}} \cite{{10.17487/RFC1591}. 
These DNS root servers are central nodes of trust and failure, and cyber attack such as DDoS may leads to the whole system taken down. 