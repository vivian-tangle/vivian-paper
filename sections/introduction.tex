\section{Introduction}
\label{sec:introduction}

A distributed ledger is a type of distributed database that assumes the presence of nodes with malicious intentions.
And distributed ledger technology (DLT) enables the realization and operation of distributed ledgers,
which allows benign nodes, through a shared consensus mechanism, to agree on an almost immutable record of transactions with Byzantine failure tolerance (BFT) and eventual consistency \cite{Sunyaev2020}.
Blockchain is one of the most well-known DLTs which was first implemented on Bitcoin. It proposed a simple but robust way for transaction data storage without relying on trust of third parties \cite{nakamoto2008peer}.
Blockchain also ensures improved security and anonymity of Bitcoin transactions compared with traditional electronic transactions.
Since the introduction of Bitcoin in 2009, cryptocurrencies based on DLTs have made a great impact on financial sectors. People also discovered that the usefulness of DLTs is beyond exchange of currencies and
significant adoption of DLTs were made in many other industries for other different services later on. Namecoin is the first altcoin (any cryptocurrencies that are not Bitcoin) for being the first to create its own blockchain separate from Bitcoin's \cite{kalodner2015empirical}.
The creation of Namecoin was inspired by the idea of BitDNS \cite{merited2010bitdns} and for establishing a decentralized domain name looking up system.
