\section{Conclusion}
\label{sec:conclusion}

Current Internet services based on centralized data models have potential risk of single point of failure. 
And data storage relies on the trust of third parties such as service providers. This leads to user data integrity and privacy issues.
In this paper, we introduce Vivian, a new decentralized global naming and storage system, which is a possible solution to the problems above.
Vivian squares Zooko's Triangle trilemma and provides a decentralized naming system, that allows users to register human-meaningful names with binding information.
Its storage layer also helps users to save their files in a decentralized and secure way. 
Unlike other blockchain based decentralized naming system, Vivian uses IOTA Tangle DL for critical data binding. 
Tangle is a lightweight and highly scalable DAG distributed ledger, which allows devices with poor computing power to write and send transactions on it.
It extends Vivian's usage in IoT networks.
In addition, the system is more environmentally friendly compared with other PoW blockchain applications as it does not require miners to do Proof-of-Work computations.
We hope the design of Vivian can help inspire more innovations of DLT applications to make more impacts in different industries.
